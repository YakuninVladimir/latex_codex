%моя любимая преамбула
\documentclass[12pt]{article} %документ с 12 шрифтом
\usepackage{blindtext} %чтобы делать lorem ipsum
\usepackage{xcolor} %цветной текст
\usepackage[a4paper, left = 2cm, right = 2cm, top = 1cm]{geometry}%геометрия документа (отступы короче)
\usepackage{graphicx} % Required for inserting images
\usepackage[utf8x]{inputenc} % Включаем поддержку UTF8  
\usepackage[russian]{babel} %поддержка русского языка
\usepackage{amsmath} %пакеты для некоторых математических символов
\usepackage{ amssymb }
\usepackage{ dsfont } %красивые буковки с двойными черточками, как для множества R
\usepackage{physics} %и снова символы (тех считает, что самые красивые греческие буквы нужны физикам)
\usepackage[normalem]{ulem}


\begin{document}

\begin{titlepage}
    \begin{center}
  
       \vspace*{1cm}
        \huge
       \textbf{Вопросики по линальчику}
        \normalsize

  
            
       \vspace{1.5cm}

       \begin{flushright}
       Сделали n неравнодушных шизов:\\
       Владимир Якунин \\
       (waiting for answer)Надежда Белоусова\\
       (waiting for commits) Ирина клочкова\\
       (waiting for commits) Арсений Поздняков\\
       \end{flushright}

       \vfill
            тут могла быть ваша реклама
       \vfill
     
        если у вас возникли вопросики, пишите сюда YakuninVla@yandex.ru
        
       МГУ, Факультет ВМиК\\
       Россия\\
       Май 2023
    \end{center}
\end{titlepage}

\tableofcontents

\section{Линейные операторы, инвариантность, спектр}

\begin{enumerate}
    \item \textbf{Докажите, что линейная комбинация линейных операторов является линейным оператором.}
    \item \textbf{Докажите, что произведение операторов является линейным оператором.}
    \item \textbf{Докажите, что если линейный оператор обратим, то обратный к нему определён однозначно и является линейным оператором.}
    \item \textbf{Пусть $V$ и $W$ --- конечномерные пространства над общим полем. Докажите, что для обратимости линейного оператора $A: V \rightarrow W$ необходимо и достаточно выполнение условий $dimV = dimW$ и $kerA = 0$.}
    \item \textbf{Докажите, что ядро и образ линейного оператора являются его инвариантными подпространствами.}
    \item \textbf{Докажите, что сумма ранга и дефекта линейного оператора равна размерности его области определения.}
    \item \textbf{Докажите, что сумма ядер двух линейных операторов, действующих на одном пространстве, совпадает с этим пространством, то образ суммы равен сумме образов.}
    \item \textbf{Линейный оператор $A: V \rightarrow V$ удовлетворяет равенству $A^m = 0$. Докажите, что оператор $I - A$ --- обратим.}
    \item \textbf{Линейные операторы $A$ и $B$ таковы, что оператор $A + B$ обратимый. Докажите, что операторы $P = (A + B)^{-1}A$ и $Q = (A + B)^{-1}B$ коммутируют.}
    \item \textbf{Докажите, что для того, чтобы оператор $P: V \rightarrow V$ был оператором проектирования, необходимо и достаточно, чтобы выполнялось условие $P^2 = P$.}
    \item \textbf{Докажите, что для того чтобы матрицы одного размера были матрицами одного и того же линейного оператора в каких-то арах базисов необходимо и достаточно, чтобы они имели одинаковый ранг.}
    \item \textbf{Докажите, что определитель и след квадратной матрицы являются инвариантами подобия}
    \item \textbf{Докажите, что барактеристический многочлен квадратной матрицы является инвариантом подобия.}
    \item \textbf{Найдите характеристический многочлен матрицы $A = 
    \begin{bmatrix}
         &&1\\
         &\rotatebox[origin=c]{75}{$\ddots$}&\\
         1&& \\
    \end{bmatrix}_{n \times n}
    $}.
    \item \textbf{Найдите все инвариантные подпространства оператора дифференциирования в пространстве всех вещественных многочленов.}
    \item \textbf{Докажите, что число является собственным значением линейного оператора на конечномерном пространстве в том и только в том случае, когда оно является корнем его характеристического многочлена.}
    \item \textbf{линейный оператор действует в $n$-мерном пространстве над полем, содержащем все корни его характеристического многочлена. Докажите, что существует базис пространства , в котором матрица оператора имеет верхний треугольный вид с главной диагональю, заполненной корнями характеристического многочлена в любом заранее заданном порядке.}
    \item \textbf{Докажите, что если матрицы $A$ и $B$ подобны, то для произвольного многочлена $f(\lambda)$ матрицы $f(A)$ и $f(B)$ тоже подобны.}
    \item \textbf{Докажите, что минимальный многочлен, аннулирующий квадратную матрицу, является делителем её характеристического многочлена.}
    \item \textbf{докажите, что любая квадратная матрица с элементами из произвольного поля аннулируется своим характеристическим многочленом.}
    \item \textbf{Докажите, что любой приведённый многочлен степени выше первой является характеристическим многочленом некоторой матрицы.}
    \item \textbf{Преобразование $A \rightarrow PAP^{-1}$ будем называть элементарным преобразованием подобия, если матрица $P$ является матрицей перестановки, либо матрицей вида $P = I + \gamma E_{ij}$, где матрица $E_{ij} $ отличается от нулевой только единицей в позиции $(i, j)$, при $i \neq j$, а число $\gamma$ --- произвольное. Докажите, что с помощью $O(n^2)$ элементарных преобразований подобия матрицу порядка $n$ можно привести к верхнему почти треугольному виду}%ура, путём долгих раздумий я понял, почему не просто к треугольному виду, там же может быть верхняя строка из нулей.
    \item \textbf{Докажите, что алгебраическая кратность собственного значения не меньше его геометрической кратности.}
    \item \textbf{Докажите, что собственные векторы попарно различных собственных значений линейно независимы.}
    \item \textbf{докажите, что если матрица порядка $n$ имеет $n$ собственных значений, то она диагонализуема.}
    \item \textbf{Докажите, что для нильпотентности линейного оператора необходимо и достаточно, чтобы он был квазискалярным, с собственным значением $0$.}
    \item \textbf{Докажите, что линейный оператор  $A$ является квазискалярным с елинственным собственным значением $\lambda$, если сдвинутый оператор $A - \lambda I$ является нильпотентным.}
    \item \textbf{Докажите, что любой вырожденный оператор либо является нильпотентным, либо расщепляется в сумму нильпотентного и обратимого операторов.}
    \item \textbf{Пусть линейный оператор действует на конечномерном пространстве над полем, которое содержит все корни его характеристического многочлена. докажите, что он расщепляется на прямую сумму своих сужений на корневые подпространства своих попарно различных собственных значений.}
    \item \textbf{Матрица $A$ порядка $n$ имеет попарно различные собственные значения $\lambda_1, \cdots ,\lambda_n$ и соответствующие им собственные векторы $v_1, \cdots, v_n $. Найти собственные векторы линейного оператора $X \rightarrow A^3XA^4, \quad X \in \mathds{C}^{n \times n}$.}
    \item \textbf{Докажите, что минимальное инвариантное относительно оператора $A$ подпространство $M(A, x)$, содержащее заданный ненулевой вектор $x$, совпадает с пространством Крылова $L_k(A, x)$, содержащим вектор $A^kx$. Его размерность равна минимальному значению $k$, при котором $A^kx \in L_k(A, x)$}
    \item \textbf{Докажите, что минимальное инвариантное подпространство относительно оператора $A$ --- $M(A, x)$, содержащее заданный ненулевой вектор $x$, нерасщепляемо тогда и только тогда, когда сужение оператора $A$ на нём квазискалярно.}
    \item \textbf{Докажите, что если $B$ --- нильпотентный оператор, векторы $x, Bx, \cdots, B^{k - 1}x$ ненулевые, а $B^kx = 0$, то $x, Bx, \cdots, B^{k - 1}x$ векторы линейно независимы.}
    \item \textbf{Линейный оператор $A$ называется нильпотентным на векторе $x \neq 0$, если существует натуральное число $k$, для которого $A^kx = 0$. Минимальное такое $k$ называется индексом нильпотентности оператора $A$  на векторе $x$. Пусть $A$ --- линейный оператор и $k_1, \cdots, k_t$ --- его индексы нильпотентности на векторах $x_1, \cdots, x_t$. Докажите, что для линейной независимости составной системы векторов Крылова $x_1, Ax_1, \cdots, A^{k_1 - 1}x_1, \cdots , x_t, Ax_t, \cdots, A^{k_t - 1}x_t$ необходима и достаточна линейная независимость векторов $A^{k_1 - 1}x_1, \cdots, A^{k_t - 1}x_t$.}
    \item \textbf{докажите, что любой нильпотентный оператор, действующий на конечномерном пространстве, расщепляется в прямую сумму нерасщепляемых операторов, действущих на инвариантных подпространствах Крылова.}
    \item \textbf{Докажите, что в любом расщеплении конечномерного пространства в прямую сумму инвариантных подпространств Крылова для нильпотентного оператора $A$ число подпространств размерности $k$ равно $N^k$}
\end{enumerate}
\section{Расстояния, нормы, скалярные произведения, полиэдры}
\begin{enumerate}%задачки оформляем так
    \item \textbf{текст очередной задачи}\\

    Решение/доказательство
    \item \textbf{текст ещё одной задачи}%и так далее
\end{enumerate}
\section{Нормальные операторы, эрмитовы матрицы}
\begin{enumerate}
    \item 
\end{enumerate}

\end{document}
