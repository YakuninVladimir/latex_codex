%моя любимая преамбула
\documentclass[12pt]{article} %документ с 12 шрифтом
\usepackage{blindtext} %чтобы делать lorem ipsum
\usepackage{xcolor} %цветной текст
\usepackage[a4paper, left = 2cm, right = 2cm, top = 1cm]{geometry}%геометрия документа (отступы короче)
\usepackage{graphicx} % Required for inserting images
\usepackage[utf8x]{inputenc} % Включаем поддержку UTF8  
\usepackage[russian]{babel} %поддержка русского языка
\usepackage{amsmath} %пакеты для некоторых математических символов
\usepackage{ amssymb }
\usepackage{ dsfont } %красивые буковки с двойными черточками, как для множества R
\usepackage{physics} %и снова символы (тех считает, что самые красивые греческие буквы нужны физикам)
\usepackage[normalem]{ulem}


\begin{document}

\begin{titlepage}
    \begin{center}
  
       \vspace*{1cm}
        \huge
       \textbf{Вопросики по линальчику}
        \normalsize

  
            
       \vspace{1.5cm}

       \begin{flushright}
       Сделали n неравнодушных шизов:\\
       Владимир Якунин \\
       (waiting for answer)Надежда Белоусова\\
       (waiting for commits) Ирина клочкова\\
       (waiting for commits) Арсений Поздняков\\
       \end{flushright}

       \vfill
            тут могла быть ваша реклама
       \vfill
     
        если у вас возникли вопросики, пишите сюда YakuninVla@yandex.ru
        
       МГУ, Факультет ВМиК\\
       Россия\\
       Май 2023
    \end{center}
\end{titlepage}

\tableofcontents

\section{Линейные операторы, инвариантность, спектр}

\begin{enumerate}
    \item \textbf{Докажите, что линейная комбинация линейных операторов является линейным оператором.}\\
    
    Рассмотрим оператор $A = \gamma_1 A_1 + \cdots + \gamma_n A_n$ и проверим, является ли он линейным:
    \[
    A(\alpha x + \beta y) = (\gamma_1 A_1 + \cdots + \gamma_n A_n)(\alpha x + \beta y) = \gamma_1 A_1 \alpha x + \cdots + \gamma_n A_n \alpha x + \gamma_1 A_1 \beta y + \cdots + \gamma_n A_n \beta y = \]\[
    = \alpha \gamma_1 A_1  x + \cdots + \alpha \gamma_n A_n  x + \beta \gamma_1 A_1  y + \cdots + \beta \gamma_n A_n  y =\]\[ \alpha(\gamma_1 A_1 + \cdots + \gamma_n A_n)x + \beta(\gamma_1 A_1 + \cdots + \gamma_n A_n)y = \alpha A x + \beta A y
    \]
    таким образом оператор $A$ --- линейный, чтд.
    \item \textbf{Докажите, что произведение линейных операторов является линейным оператором.}\\
    
     Пусть есть линейные операторы $A: V \rightarrow W$ и $B : W \rightarrow Z$ .Рассмотрим оператор $C : V \rightarrow Z$, $C = BA$ и проверим, является ли он линейным:
     \[
     C(\alpha x + \beta y) = BA(\alpha x + \beta y) = BA\alpha x + BA \beta y = B (\alpha A x) + B(\beta A  y) = \alpha BA x + \beta AB y = \alpha C x + \beta C y
     \]
     как видим, оператор $C$ --- линейный.
    \item \textbf{Докажите, что если линейный оператор обратим, то обратный к нему определён однозначно и является линейным оператором.}\\
    
    пусть $A : V \rightarrow W$ --- обратимый линейный оператор и $B : W \rightarrow V$ --- обратный оператор к $A$: $BA = I = ABAB = AB$.  \\
    Единственность: пусть есть оператор $C \neq B$ такой что $AC = CA = I$, тогда 
    \[
    0 = I - I = AB - AC = A(B - C)
    \]
    значит либо $A = 0$, либо $B - C = 0 \Rightarrow B = C$.\\
    Линейность: \[
    B(\alpha x + \beta y) = B(\alpha AB x + \beta AB y) = B(A(\alpha B x + \beta B y)) = \alpha B x + \beta B y
    \]
    несложно увидеть, что оператор $B$ --- линейный. дальше обозначаем по-нормальному $B = A^{-1}$.
    
    \item \textbf{Пусть $V$ и $W$ --- конечномерные пространства над общим полем. Докажите, что для обратимости линейного оператора $A: V \rightarrow W$ необходимо и достаточно выполнение условий $dimV = dimW$ и $kerA = 0$.}\\

    Необходимость: Так как для $\forall x \in V$ существует прообраз $y = Ax \in W$, ведь $x = A^{-1}y$, то $dimV \leqslant dimW$. Так как для $\forall y \in W$ существует прообраз $x = A^{-1}y \in V$, ведь $y = Ax$, то $dimW \leqslant dimV$. А значит $dimV = dimW$. Теперь пусть существует $x \in V, \quad x \neq 0$ принадлежащий ядру оператора $A$: $Ax = 0$. Тогда $A^{-1}0 = x$, но так как $A^{-1}$ --- линейный, то $A^{-1}0 = 0 \neq x$ --- противоречие.\\

    Достаточность: пусть $e_1, \cdots, e_n$ --- базис $V$, тогда если $x = \alpha_1 e_1 + \cdots + \alpha_n e_n$, то $y = Ax = \alpha_1 Ae_1 + \cdots + \alpha_n Ae_n$, если вектора $Ae_1, \cdots, Ae_n$ --- линейно зависимы, то существует их нетривиальная линейная комбинация, равная нулю: $\beta_1Ae_1 + \cdots + \beta_nAe_n = 0 \Rightarrow A(\beta_1 e_1 + \cdots + \beta_n e_n) = 0$, однако так как $e_1, \cdots, e_n$ линейно независимы, то $z = \beta_1 e_1 + \cdots + \beta_n e_n \neq 0$, но это значит, что $\exists z \in V, z \neq 0 : Az = 0 \Rightarrow kerA \neq 0$ --- противоречие, значит $Ae_1, \cdots, Ae_n$ линейно независимы и их число равно размерности $W$, а значит это базис $W$.\\
    
    Теперь определим действие оператора $A^{-1}$ на базисных векторах $W$ и покажем, что он будет обратным к $A$:
    $A^{-1}(Ae_k) = e_k \quad \forall k \in \{1, \cdots, n\}$ такой оператор существует и единственный (потому что для того, чтобы задать оператор, достаточно задать его действие на базисных векторах области определения), и при этом $A^{-1}Ax = A^{-1}(\alpha_1 Ae_1 + \cdots + \alpha_n Ae_n) = \alpha_1 A^{-1} Ae_1 + \cdots + \alpha_n A^{-1} Ae_n = \alpha_1 e_1 + \cdots + \alpha_n e_n = x$, значит $A^{-1}$ --- обратный оператор. Ч.Т.Д.
    \item \textbf{Докажите, что ядро и образ линейного оператора являются его инвариантными подпространствами.}\\

    \textit{Ядро}: Пусть $A: V \rightarrow V$, $x \in  V,\quad Ax = 0$, то есть $x \in kerA$, тогда $Ax = 0 \in kerA \quad \forall x \in kerA$\\

    \textit{образ}: Пусть $A: V \rightarrow V$, $y \in imA$, то есть $\exists x: Ax = y$, тогда $Ay = z \in imA$, так как у $z$ есть прообраз $y$.
    
    \item \textbf{Докажите, что сумма ранга и дефекта линейного оператора равна размерности его области определения.}\\

    Ядро линейного оператора $A: V \rightarrow W$ --- линейное подпространство пространства $V$, базис ядра: $a_1, \cdots, a_k$, размерность ядра $dim(kerA) = k$. дополним его до базиса пространства $V$: $a_1, \cdots, a_k, b_{k + 1}, \cdots, b_n$. Заметим, что так как $b_{k + 1}, \cdots, b_n \in V \setminus kerA$, то $Ab_i \neq 0 \forall i \in \{k + 1, \cdots, n\}$, вектора $Ab_{k + 1}, \cdots, Ab_n$ --- линейно независимы.\\
    
    Действительно, если бы они были бы линейно зависимы, то существовала бы нетривиальная линейная комбинация $\alpha_1 Ab_{k + 1} + \cdots + \alpha_{n - k} A b_n = 0 \Rightarrow A(\alpha_1 b_{k + 1} + \cdots + \alpha_{n - k} b_n) = 0$, то есть $\alpha_1 b_{k + 1} + \cdots + \alpha_{n - k} b_n \in kerA$, а это протиоречит выбору векторов $b_{k + 1}, \cdots, b_n$.\\
    
    Теперь разложим $\forall x \in V$ в выбранном базисе, подействуем на него оператором и посмотрим, что интересного произойдёт: $Ax = \alpha_1 Aa_1 + \cdots + \alpha_k Aa_k + \alpha_{k + 1}Ab_{k + 1} + \cdots + \alpha_nAb_n = \alpha_{k + 1}Ab_{k + 1} + \cdots + \alpha_nAb_n$, заметим, что любой вектор из образа можно представить как линейную комбинацию $n - k$ линейно независимых векторов $Ab_{k + 1} + \cdots + Ab_n$, то есть это базис образа, то есть $dim(imA) = n - k$.\\
    
    Осталось сложить дважды два: $dim(kerA) + dimA(imA) = k + (n - k) = n$. Ч.Т.Д.
    A
    \item \textbf{Докажите, что если сумма ядер двух линейных операторов, действующих на одном пространстве, совпадает с этим пространством, то образ суммы равен сумме образов.}\\

    Дано: $A: V \rightarrow V, B: V \rightarrow V$, $kerA + kerB = V$, это означает, что любой вектор $\forall x \in V$ можно представить в виде $x = a + b$, где $a \in kerA \quad b \in kerB$, тогда подействуем на этот вектор оператором $A + B$: $(A + B)x = (A + B)(a + b) = Aa + Ab + Ba + Bb = Ab + Ba$, где $Ab \in imA, Ba \in imB$. То есть любой вектор из образа суммы лежит в сумме образов.\\
    
    Теперь рассмотрим вектор $x = \alpha Ay + \beta Bz$ --- вектор из суммы образов. Для удобства сделаем замену и избавимся от коэффициентов: $\alpha Ay = A(\alpha y) = Ay'$, $\beta B z = B(\beta z) = Bz'$ . Представим вектора $y' = a_1 + b_1$, $z' = a_2 + b_2$, где $a_1, a_2 \in kerA$, $b_1, b_2 \in kerB$, тогда $x = Ay' + Bz' = Aa_1 + Ab_1 + Ba_2 + Bb_2 = Ab_1 + Ba_2 = Aa_2 + Ab_1 + Ba_2 + Bb_1 = (A + B)(a_2 + b_1)$. То есть любой вектор из суммы образов является вектором из образа суммы.\\

    Таким образом мы доказали, что $x \in im(A +B) \Rightarrow x \in imA + imB$ и $x \in imA + imB \Rightarrow x \in im(A +B)$, то есть $im(A + B) = imA + imB$. Ч.Т.Д.
    
    \item \textbf{Линейный оператор $A: V \rightarrow V$ удовлетворяет равенству $A^m = 0$. Докажите, что оператор $I - A$ --- обратим.}\\

    В данной задаче требуется догадаться и вспомнить школьную формулу: $x^n - y^n = (x - y)(x^{n - 1} + x^{n - 2}y + \cdots + y^{n - 1})$.\\
    По этой формулке $I = I - A^m = (I - A)(I + A + A^2 + \cdots + A^{m - })$.\\
    Из этого следует, что $I - A$ обратим и обратный к нему  $A^{-1} = I + A + A^2 + \cdots + A^{m - 1}$.\\
    
    \item \textbf{Линейные операторы $A$ и $B$ таковы, что оператор $A + B$ обратимый. Докажите, что операторы $P = (A + B)^{-1}A$ и $Q = (A + B)^{-1}B$ коммутируют.}\\

    Рассмотрим сумму операторов $P$ и $Q$: $P + Q = (A + B)^{-1}A + (A + B)^{-1}B = (A + B)^{-1}(A + B) = I \Rightarrow P = I - Q$. А значит $PQ = (I - Q)Q = Q - Q^2 = Q(I - Q) = QP$. Ч.Т.Д.\\
    
    \item \textbf{Докажите, что для того, чтобы оператор $P: V \rightarrow V$ был оператором проектирования, необходимо и достаточно, чтобы выполнялось условие $P^2 = P$.}\\

    \textit{Необходимость:} $V = \L_1 \oplus L_2$, $P : V \rightarrow V$ --- оператор проектирования на $L_1$ параллельно $L_2$, то есть для $\forall x \in V x = x_1 + x_2$, где $x_1 \in L_1, x_2 \in L_2$ вектор $Px = x_1$, покажем, что тогда $P^2 = P$: $\forall x \in V x = x_1 + x_2$, где $x_1 \in L_1, x_2 \in L_2$ вектор $P^2x = P(Px) = Px_1 = x_1$, то есть $P^2 = P$.\\

    \textit{Достаточность:} Оператор $P : V \rightarrow V$ такой, что $P^2 = P$, докажем, что он является оператором проектирования: рассмотрим какой-то рандомный вектор $y \in V$ и подействуем на него оператором $Py = x \in imP$, и ещё раз подействуем $P^2y = P(Py) = Px$, так как $P^2 = P$, то $Px = x = Py$, теперь введём вектор $z = y - x$, тогда $y = x + z$, подействуем на это выражение оператором $P$: $Py = Px + Pz \Rightarrow x = x + Pz \Rightarrow Pz = 0 \Rightarrow z \in kerP$.\\
    Заметим, что вектора $x$ и $z$ определены однозначно, то есть $V = imP \oplus kerP$ и  для $\forall x \in V x = x_1 + x_2$, где $x_1 \in imP$ и $x_2 \in kerP$ выполнено $Px = x_1$, то есть $p$ --- оператор проектирования. Ч.Т.Д.
    
    \item \textbf{Докажите, что для того чтобы матрицы одного размера были матрицами одного и того же линейного оператора в каких-то парах базисов необходимо и достаточно, чтобы они имели одинаковый ранг.}\\

    \begin{center}
        ВНИМАНИЕ!\\
        Тема перехода из одного базиса в другой зачастую вызывает непроизвольное воспламенение в области мозга и не только, потому что $Q_{eg}$ у Тыртышникова это матрица перехода из $g$ в $e$, а в задачнике Ким это матрица перехода из $e$ в $g$. Из-за этого, очевидно, возникает нехилая путаница, поэтому тут я буду придерживаться обозначений Евгения Евгеньевича во избежание недопонимания.
    \end{center}

    По Тыртышу матрицей перехода из старого базиса $g$ в новый базис $e$ называется матрица $Q_{eg}$ столбцы которой --- координаты векторов старого базиса $g$ в новом базисе $e$. Вектора в разных базисах связаны следующим соотношением: $[x]_e = Q_{eg}[x]_g$.\\
    
    Общая формула для матрицы оператора при переходе от одной пары базисов в другую для оператора $A: V \rightarrow W$: $[A]_{hg} = Q_{hf}[A]_{fe}P_{eg}$, где $h$ и $f$ --- базисы $W$, а $e$ и $g$ --- базисы $V$.\\

    Если оператор $A: V \rightarrow V$, то это соотношение можно переписать как $[A]_g = Q_{eg}^{-1}A_eQ_{eg}$

    \textit{Решение:} Пусть есть 2 матрицы операторов $A_{m \times n}$ и $B_{m \times n}$, их можно элементарными преобразованиями привести к диагональному виду, где первые $r_1 = rankA$ ($r_2 = rankB$ соответственно) элементов на диагонали --- единицы, а остальные нули:
    \[
    A = Q_1
    \begin{bmatrix}
        I_{r_1 \times r_1} & 0 \\
        0 & 0
    \end{bmatrix}P_1
    \quad
    B = Q_2
    \begin{bmatrix}
        I_{r_2 \times r_2} & 0 \\
        0 & 0
    \end{bmatrix}P_2
    \]
    При этом матрицы $Q_1, Q_2, P_1, P_2$ невырожденные. Так как матрица перехода из одного базиса в другой всегда невырожденная, а умножение на невырожденную матрицу не меняет ранг, то равенство рангов является необходимым условием для того, чтобы матрицы $A$ и $B$ были матрицами одного и того же оператора. Если же $r_1 = r_2$, то так как матрицы перехода невырожденны, то:
    \[
    \begin{bmatrix}
        I_{r_2 \times r_2} & 0 \\
        0 & 0
    \end{bmatrix} = Q_2^{-1}BP_2^{-1} 
    \Rightarrow A = Q_1Q_2^{-1}BP_2^{-1}P_1 = QBP
    \]
    а значит существуют такие матрицы перехода от одной пары базисов к другому $Q$ и $P$, что в первой паре базисов оператор имеет матрицу $A$, а во второй $B$, то есть равенство рангов это достаточное условие. Ч.Т.Д.
    \item \textbf{Докажите, что определитель и след квадратной матрицы являются инвариантами подобия}

    \textit{Определитель:} Матрица $B$ подобна матрице $A$, то есть $B = Q^{-1}AQ \Rightarrow |B| = |Q^{-1}||A||Q| = |Q^{-1}||Q||A| = |Q^{-1}Q||A| = |I||A| = |A|$ так как произведение определителей равно определителю произведения. Ч.Т.Д.\\

    \textit{След:} Легко заметить, что $tr(AB) = \sum \limits_{1 \leqslant i, j \leqslant n} a_{ij}b_{ji}$, из этого очевидно, что $tr(AB) = tr(BA)$, а значит $tr(B) = tr(Q^{-1}AQ) = tr(Q^{-1}QA) = tr(A)$. Ч.Т.Д.
    
    \item \textbf{Докажите, что характеристический многочлен квадратной матрицы является инвариантом подобия.}\\
    Матрица $B$ подобна матрице $A$, то есть $B = Q^{-1}AQ$, тогда найдём определитель матрицы $B - \lambda I$: $|B - \lambda I| = |Q^{-1}AQ - \lambda I| = |Q^{-1}AQ - \lambda Q^{}-1IQ| = |Q^{-1}(A - \lambda I)Q| = |Q^{-1}||A - \lambda I||Q| = |Q^{-1}||Q||A - \lambda I| = |Q^{-1}Q||A - \lambda I| = |A - \lambda I|$. Как видим, характеристические многочлены совпадают. Ч.Т.Д.
    
    \item \textbf{Найдите характеристический многочлен матрицы $A = 
    \begin{bmatrix}
         &&1\\
         &\rotatebox[origin=c]{75}{$\ddots$}&\\
         1&& \\
    \end{bmatrix}_{n \times n}
    $}.\\

    Запишем матричку $A - \lambda I$ и найдём её характеристический многочлен разложив по Лапласу последнюю строку: 
    \[
    |A - \lambda I| = 
    \begin{vmatrix}
        -\lambda & 0 & \cdots & 0 & 1\\
        0 & -\lambda & \cdots & 1 & 0\\
        \vdots & \vdots & \ddots & \vdots & \vdots \\
        0 & 1 & \cdots & -\lambda & 0\\
        1 & 0 & \cdots & 0 & -\lambda\\
    \end{vmatrix} = 
    -\lambda 
    \begin{vmatrix}
        -\lambda & 0 & \cdots & 0 \\
        0 & -\lambda & \cdots & 1 \\
        \vdots & \vdots & \ddots & \vdots  \\
        0 & 1 & \cdots & -\lambda \\
    \end{vmatrix}
     + (-1)^{n - 1}
     \begin{vmatrix}
          0 & \cdots & 0 & 1\\
         -\lambda & \cdots & 1 & 0\\
         \vdots & \ddots & \vdots & \vdots \\
         1 & \cdots & -\lambda & 0\\
     \end{vmatrix}
    \]
    Будем обозначать определитель $|A - \lambda I|, A_{k \times k}$ как $D_k$, тогда искомый определитель это $D_n$, теперь то, что мы получили на предыдущем шаге ещё раз разложим по Лапласу по первой строке и выразим $D_n$ через $D_{n - 2}$:
    \[
    D_n = (-\lambda)(-\lambda)
    \begin{vmatrix}
        -\lambda & \cdots & 1\\
        \vdots & \ddots & \vdots \\
        1 & \cdots & -\lambda\\
    \end{vmatrix}
     + (-1)^{n - 1}(-1)^{n - 2}
     \begin{vmatrix}
        -\lambda & \cdots & 1\\
        \vdots & \ddots & \vdots \\
        1 & \cdots & -\lambda\\
    \end{vmatrix}
     = \lambda^2 D_{n - 2} - D_{n - 2} = (\lambda^2 - 1)D_{n - 2}
    \]
    это разложение верно для $n > 2$, найдём $D_1$ и $D_2$: $D_1 = |1 - \lambda| = 1 - \lambda$,\\ $D_2 = 
    \begin{vmatrix}
        -\lambda & 1\\
        1 & -\lambda\\
    \end{vmatrix}
     = \lambda^2 - 1
    $
    тогда если $n$ чётное, то $D_n = (\lambda^2 - 1)^{\frac{n}{2}}$, а если нечётное, то $D_n = (\lambda^2 - 1)^{\frac{n - 1}{2}}(1 - \lambda)$.\\
    Ответ:\\ при $n = 2k, k \in \mathds{R}$ характеристический многочлен равен $|A - \lambda I| = (\lambda^2 - 1)^{\frac{n}{2}}$\\
    при $n = 2k - 1, k \in \mathds{R}$ этот многочлен равен $|A - \lambda I| = (\lambda^2 - 1)^{\frac{n - 1}{2}}(1 - \lambda)$.\\
    \item \textbf{Найдите все инвариантные подпространства оператора дифференциирования в пространстве всех вещественных многочленов.}\\

    Возьмём многочлен $f(t)$ степени $n$ и найдём минимальное инвариантное подпространство $L$, в котором он содержится. Заметим, что в этом пространстве должны содержаться также многочлены $Af, A^2f, \cdots, A^nf$ степеней $n - 1, n - 2, \cdots, 0$, а также все их линейные комбинации. Запишем все многочлены $f, Af, A^2f, \cdots, A^nf$ в виде матрицы, где строки это коэффициенты соответствующего многочлена, степени записаны начиная с $n$-ной слева направо:\\
    \[
    \begin{bmatrix}
        a_n & a_{n - 1} & \cdots & a_1 & a_0\\
        0 & na_n & \cdots & 2a_2 & a_1\\
        \vdots & \vdots & \ddots & \vdots & \vdots\\
        0 & 0 &\cdots & n!a_n & (n - 1)!a_{n -1}\\
        0 & 0 &\cdots & 0 & n!a_n\\
    \end{bmatrix} \rightarrow
    \begin{bmatrix}
        a_n & 0 & \cdots & 0 & 0\\
        0 & na_n & \cdots & 0 & 0\\
        \vdots & \vdots & \ddots & \vdots & \vdots\\
        0 & 0 &\cdots & n!a_n & 0\\
        0 & 0 &\cdots & 0 & n!a_n\\
    \end{bmatrix} \rightarrow
    \begin{bmatrix}
        1 & 0 & \cdots & 0 & 0\\
        0 & 1 & \cdots & 0 & 0\\
        \vdots & \vdots & \ddots & \vdots & \vdots\\
        0 & 0 &\cdots & 1 & 0\\
        0 & 0 &\cdots & 0 & 1\\
    \end{bmatrix}
    \]
    То есть $L$ обязано содержать многочлены $1, t, \cdots, t^n$ и их линейные комбинации, то есть $L \supset M_n$ где $M_n$ --- множество всех многочленов степени не более $n$, при этом $M_n$ является инвариантным относительно оператора дифференциирования $D$, так как он только понижает степень многочлена. То есть $L = M_n$. \\

    Теперь пусть у нас есть линейное подпространство $Z$, найдём минимальное инвариантное подпространство $L \supset Z$. Пусть $n$ --- максиманьная степень многочлена из $Z$, тогда очевидно, что$Z \subset M_n$, причём $M_n$ --- минимальное такое подпространство.\\

    Таким образом как бы мы не выбирали подпространство пространства многочленов минимальное инвариантное пространство, содержащее его это $M_n$ и пространство само является инвариантным только если оно равно $M_n$.\\

    Ответ: инвариантными будут все подпространства $M_n, n \in \mathds{N} + {0}$ и никакие другие.
    \item \textbf{Докажите, что число является собственным значением линейного оператора на конечномерном пространстве в том и только в том случае, когда оно является корнем его характеристического многочлена.}\\

    $\Rightarrow$ Пусть $\lambda$ --- собственное значение оператора $A: V \rightarrow V$, то есть $Ax = \lambda x$ для некоторого ненулевого $x \in V$, перенесём всё в левую часть: $Ax - \lambda x = 0 \Leftrightarrow Ax - \lambda I x = 0 \Leftrightarrow (A - \lambda I)x = 0$, а это означает, что существует нетривиальная линейная комбинация столбцов, равная нулю, а значит они линейно зависимы, а это означает, что $|A - \lambda I| = 0$, что и означает, что $\lambda$ --- корень характеристического многочлена линейного оператора $A$.\\

    $\Leftarrow$ Пусть $\lambda$ --- корень характеристического многочлена, то есть $|A - \lambda I| = 0$, это означает, что столбцы матрицы $A - \lambda I$ линейно зависимы, а значит существует ненулевой вектор $x$ такой, что $(A - \lambda I)x = 0 \Leftrightarrow Ax - \lambda x = 0 \Leftrightarrow Ax = \lambda x$, то есть $\lambda$ --- собственное значение линейного оператора $A$. Ч.Т.Д.
    
    \item \textbf{линейный оператор действует в $n$-мерном пространстве над полем, содержащем все корни его характеристического многочлена. Докажите, что существует базис пространства , в котором матрица оператора имеет верхний треугольный вид с главной диагональю, заполненной корнями характеристического многочлена в любом заранее заданном порядке.}

    
    \item \textbf{Докажите, что если матрицы $A$ и $B$ подобны, то для произвольного многочлена $f(\lambda)$ матрицы $f(A)$ и $f(B)$ тоже подобны.}

    Пусть $B = Q^{-1}AQ$, а многочлен $f(\lambda) = a_0 + a_1 \lambda + \cdots + a_n \lambda^n$, тогда $f(B) = a_0 + a_1B + \cdots + a_nB^n = a_0 + a_1 Q^{-1}AQ + \cdots + a_n \underbrace{(Q^{-1}AQQ^{-1}AQ \cdots Q^{-1}AQ)}_{n} = Q^{-1}a_0Q + Q^{-1}a_0B + \cdots + Q^{-1}a_nA^nQ = Q^{-1}(a_0 + a_1A + \cdots + a_nA^n)Q = Q^{-1}f(A)Q$, то есть матрицы $f(B)$ и $f(A)$ подобны. Ч.Т.Д.\\
    
    \item \textbf{Докажите, что минимальный многочлен, аннулирующий квадратную матрицу, является делителем её характеристического многочлена.}\\

    Доказательство от противного. Из теоремы Кэли-Гамильтона известно, что характеристический многочлен $f(\lambda)$ является аннулирующим для матрицы $A$, пусть $g(\lambda)$ --- минимальный аннулиющий многочлен $A$ и он не является делителем её характеристического многочлена. Тогда поделим харантеристический многочлен на $g$ с остатком: $f(\lambda) = q(\lambda)g(\lambda) + r(\lambda)$, многочлен $r(\lambda)$ имеет меньшую степень, нежели $G(\lambda)$ и при этом $r(A) = f(A) - q(A)g(A) = 0 - Q(A)0 = 0$, то есть $r(\lambda)$ является аннулирующим многочленом, то есть $g(\lambda)$ --- не минимальный, а это противоречит нашему предположению. Значит $g(\lambda)$ является делителем характеристического многочлена. Ч.Т.Д. 
    \item \textbf{Докажите, что любая квадратная матрица с элементами из произвольного поля аннулируется своим характеристическим многочленом.}

    \begin{center}
        \textbf{Теорема Кэли-Гамильтона.}
    \end{center}
    Пусть $\mathcal{A}: V \rightarrow V$ и $A$ --- матрица линейного оператора $\mathcal{A}$, и $f(\lambda) = |A - \lambda I|$ --- характеристический многочлен $A$. Тогда $f(A) = 0$.\\

    Матричным многочленом называется выражение вида $f(\lambda) = A_0 + A_1 \lambda + \cdots + A_k \lambda^k$, а его значением при $\lambda = X$ называется $f(X) = A_0 + A_1 X + \cdots + A_k X^k$. Матричный многочлен $F(\lambda)$ можно разделить на другой матричный многочлен $G(\lambda)$ с остатком, то есть представить $F(\lambda)$ как $F(\lambda) = Q(\lambda)G(\lambda) + R(\lambda)$, причём степень $R(\lambda)$ строго меньше степени $G(\lambda)$.\\

    Для матричных многочленов существует теорема Безу: из равенства $F(\lambda) = Q(\lambda)(\lambda I - A) + R(\lambda)$ следует равенство $F(A) = R(A)$.

    \textit{Доказательство: }По индукции понижаем степень многочлена $F(\lambda) =  A_0 + A_1 \lambda + \cdots + A_k \lambda^k$: $F_1(\lambda) = F(\lambda) - A_k\lambda^{k - 1}(\lambda I - A) = F_1(\lambda) - A_k(\lambda^k I - \lambda^{k - 1}A)$, то есть $F_1(A) = F(A) - A_k(A^k - AA^{k-1}) = F(A)$, при этом многочлен $F_1(\lambda)$ имеет степень $k - 1$, аналогично можно получить многочлен $F_2(\lambda)$ степени $k - 2$ и так далее до многочлена $F_k(\lambda) = R(\lambda)$ степени $0$, при этом $F(\lambda) = R(\lambda)$.\\

    Теперь назовём $f(\lambda) = |A - \lambda I| = a_0 + a_1 \lambda + \cdots + a_n \lambda^n$, а $F(\lambda) = a_0 I + a_1 I \lambda + \cdots + a_n I \lambda^n$, при этом $F(\lambda) = I f(\lambda)$. Обозначим $B(\lambda)$ присоединённую матрицу к матрице $A - \lambda I$, то есть матрицу из алгебраических дополнений к каждому из элементов $A - \lambda I$. Для неё выполнено: $B(A - \lambda I) = f(\lambda)I  =F(\lambda)$. Применим теорему Безу и получим, что $F(A) = R(A) = 0$, так как остатка нет. Ч.Т.Д.\\


    а ещё можно было расширить поле, из которого элементы матрицы, до такого, в котором есть все корни его характеристического многочлена и диагонализовать матрицу $A$, а потом просто подставить её в характеристический многочлен.
    \item \textbf{Докажите, что любой приведённый многочлен степени выше первой является характеристическим многочленом некоторой матрицы.}\\

    \textit{Формулировка задания является не совсем корректной, так как легко заметить, что так как в матрице $A - \lambda I$ все $\lambda$ расположены по диагонали, то есть лишь одна перестановка $\sigma = \{1, 2 , \cdots, n\}$ такая, что при  $B = (A - \lambda I)$, $b_{1\sigma(1)}b_{2\sigma(2)} \cdots b_{n\sigma(n)}$ будет содержать все элементы с $\lambda$, то есть коэффициент при $\lambda^n$ будет равен ему же у $(a_{11} - \lambda)(a_{22} - \lambda)\cdots(a_{nn} - \lambda) = (-1)^n\lambda^n + \cdots$, как видим, при нечётных $n$ характеристический многочлен не может быть приведённым. Правильная формулировка:}\\

    "Докажите, что любой приведённый многочлен степени выше первой является приведенным характеристическим многочленом некоторой матрицы."

    Пусть у нас есть многочлен $g(x) = a_0 + a_1x + \cdots + a_{n - 1}x^{n - 1} + x^n$, рассмотрим следующего вида матрицу (она называется матрицей Фробениуса или сопутствующей матрицей $g(x)$): 
    \[
    F = 
    \begin{bmatrix}
        0 & 0 & 0 & \cdots & 0 &-a_0\\
        1 & 0 & 0 & \cdots & 0 &-a_1\\
        0 & 1 & 0 & \cdots & 0 & -a_2\\
        \vdots & \vdots & \vdots & \ddots & \vdots & \vdots\\
        0 & 0 & 0 & \cdots & 0 & -a_{n - 2}\\
        0 & 0 & 0 &\cdots & 1 & -a_{n - 1}\\
    \end{bmatrix}
    \]
    Вычислим определитель  матрицы $F - \lambda I$:
    \[
    f(\lambda)
    \begin{vmatrix}
        -\lambda & 0 & 0 & \cdots & 0 &-a_0\\
        1 &  -\lambda & 0 & \cdots & 0 &-a_1\\
        0 & 1 &  -\lambda & \cdots & 0 & -a_2\\
        \vdots & \vdots & \vdots & \ddots & \vdots & \vdots\\
        0 & 0 & 0 & \cdots &  -\lambda & -a_{n - 2}\\
        0 & 0 & 0 &\cdots & 1 & -a_{n - 1}  -\lambda\\
    \end{vmatrix} =\]\[ = -\lambda
    \begin{vmatrix}
     
         -\lambda & 0 & \cdots & 0 &-a_1\\
        1 &  -\lambda & \cdots & 0 & -a_2\\
        \vdots & \vdots & \ddots & \vdots & \vdots\\
         0 & 0 & \cdots &  -\lambda & -a_{n - 2}\\
         0 & 0 &\cdots & 1 & -a_{n - 1}  -\lambda\\
    \end{vmatrix} -(-1)^{n - 1}a_0 
    \begin{vmatrix}
        1 &  -\lambda & 0 & \cdots & 0 \\
        0 & 1 &  -\lambda & \cdots & 0 \\
        \vdots & \vdots & \vdots & \ddots & \vdots \\
        0 & 0 & 0 & \cdots &  -\lambda \\
        0 & 0 & 0 &\cdots & 1 \\
    \end{vmatrix} = 
    \]
    \[
    = (-1)^na_0 - \lambda
     \begin{vmatrix}
     
         -\lambda & 0 & \cdots & 0 &-a_1\\
        1 &  -\lambda & \cdots & 0 & -a_2\\
        \vdots & \vdots & \ddots & \vdots & \vdots\\
         0 & 0 & \cdots &  -\lambda & -a_{n - 2}\\
         0 & 0 &\cdots & 1 & -a_{n - 1}  -\lambda\\
    \end{vmatrix} = \]\[
    (-1)^na_0 - \lambda((-1)^{n - 1}a_1 - \lambda(\cdots -\lambda(-a_{n - 2} - \lambda(-a_{n -1} - \lambda))) = (-1)^n(a_0 + a_1\lambda + \cdots + a_{n - 1}\lambda^{n - 1} + \lambda^n) = \]\[=(-1)^ng(x)
    \]
    если привести многочлен $f(\lambda)$, то получится приведённый многочлен $g(\lambda)$, который мог быть выбран любым. Ч.Т.Д.\\
    
    \item \textbf{Преобразование $A \rightarrow PAP^{-1}$ будем называть элементарным преобразованием подобия, если матрица $P$ является матрицей перестановки, либо матрицей вида $P = I + \gamma E_{ij}$, где матрица $E_{ij} $ отличается от нулевой только единицей в позиции $(i, j)$, при $i \neq j$, а число $\gamma$ --- произвольное. Докажите, что с помощью $O(n^2)$ элементарных преобразований подобия матрицу порядка $n$ можно привести к верхнему почти треугольному виду}%ура, путём долгих раздумий я понял, почему не просто к треугольному виду, там же может быть верхняя строка из нулей.
    \item \textbf{Докажите, что алгебраическая кратность собственного значения не меньше его геометрической кратности.}
    \item \textbf{Докажите, что собственные векторы попарно различных собственных значений линейно независимы.}\\

    Доказательство от противного: предположим есть собственные вектора $x_1, \cdots,  x_k$, отвечающие различным собственным значениям $\lambda_1, \cdots, \lambda_k$: $Ax_1 = \lambda_1x_1, \cdots, Ax_k = \lambda_kx_k$. Если вектора $x_1, \cdots, x_k$ линейно зависимы, то существует нетривиальная линейная комбинация собственных векторов $\alpha_1 x_1 + \alpha_2 x_2 + \cdots + \alpha_n x_n = 0$, выразим отсюда $x_n = -\frac{\alpha_1}{\alpha_n}x_1 - \cdots - \frac{\alpha_{n - 1}}{\alpha_n}x_{n - 1} = \beta_1 x_1 + \cdots + \beta_{n - 1} x_{n - 1}$.\\

    Теперь подействуем оператором на вектор $x_n$: $Ax_n = \beta_1 A x_1 + \cdots + \beta_{n - 1} A x_{n - 1} = \beta_1 \lambda_1 x_1 + \cdots + \beta_{n - 1} \lambda_{n - 1} x_{n - 1}$, но в то же время $Ax_n = \lambda_n x_n = \beta_1 \lambda_n x_1 + \cdots + \beta_{n - 1} \lambda_n x_{n - 1}$. Приравняем эти два выражения и перенесём всё в левую часть: $\beta_1 \lambda_1 x_1 + \cdots + \beta_{n - 1} \lambda_{n - 1} x_{n - 1} = \beta_1 \lambda_n x_1 + \cdots + \beta_{n - 1} \lambda_n x_{n - 1} \Rightarrow \beta_1 ( \lambda_1 - \lambda_n) x_1 + \cdots + \beta_{n - 1}(\lambda_{n - 1} -  \lambda_n) x_{n - 1} = 0$, а так как все собственные значения $\lambda_1, \cdots, \lambda_n$ различны, то это означает, что существует нетривиальная линейная комбинация собственных векторов $x_1, \cdots, x_{n - 1}$ равная нулю, то есть если линейно зависимы вектора $x_1, \cdots, x_n$, то линейно зависимы и вектора $x_1, \cdots, x_{n - 1}$. Применяя те же рассуждения легко получить, что линейно зависимы и вектора $x_1, \cdots, x_{n - 2}$ и так далее по индукции по $n$ получаем, что линейно зависимы вектора $x_1$ и $x_2$.\\

    Это означает, что $x_2 = \alpha x_1$, при этом $Ax_1 = \lambda_1 x_1$ и $Ax_2 = \alpha Ax_1 = \alpha \lambda_1 x_1$, но $Ax_2 = \lambda_2x_2 = \alpha \lambda_2 x_1$, то есть $\alpha \lambda_1 x = \alpha \lambda_2 x$, а значит $\lambda_1 = \lambda_2$, а это противоречит исходному предположению, значит собственные вектора, отвечающие различным собственным значениям линейно независимы. Ч.Т.Д.
    
    
    \item \textbf{докажите, что если матрица порядка $n$ имеет $n$ собственных значений, то она диагонализуема.}\\

    Матрицу порядка $n$ можно рассматривать как матрицу $A$ линейного оператора $\mathcal{A}: V \rightarrow V$, где $dimV = n$. Так как $A$ имеет $n$ разлиных собственных значений, то существует $n$ линейно независимых собственных векторов $x_1, \cdots, x_n$, отвечающих различным собственным значениям, а так как размерность $V$ равна $n$, то $x_1, \cdots, x_n$ --- базис линейного пространства $V$, значит существует матрица $Q_{xe}$ перехода из старого базиса $e$ в новый базис из собственных векторов $x$, то есть $A = Q_{xe}^{-1}\Lambda Q_{xe}$, где $\Lambda$ --- матрица оператора $\mathcal{A}$ в базисе из собственных векторов.\\

    Запишем вектор из $V$ в базисе из собственных векторов, подействуем на него оператором $\mathcal{A}$ и запишем матрицу $\Lambda$ оператора $\mathcal{A}$ в новом базисе: $x = \alpha_1 x_1 + \cdots + \alpha_n x_n$, $\mathcal{A} x = \alpha_1 \mathcal{A}  x_1 + \cdots + \alpha_n \mathcal{A} x_n = \alpha_1 \lambda_1 x_1 + \cdots + \alpha_n \lambda_n x_n$, то есть просто каждая $i$-тая координата увеличилась в $\lambda_i$ раз, то есть:
    \[
    \Lambda = 
    \begin{bmatrix}
        \lambda_1 & 0 & 0 & \cdots & 0\\
        0 & \lambda_2 & 0 & \cdots & 0\\
        0 & 0 & \lambda_3 & \cdots & 0\\
        \vdots & \vdots & \vdots & \ddots & \vdots\\
        0 & 0 & 0 & \cdots & \lambda_n\\       
    \end{bmatrix}
    \]
    Так как $\Lambda = Q_{xe}AQ_{xe}^{-1}$, то это значит, что матрица $A$ диагонализуема. Ч.Т.Д.
    
    \item \textbf{Докажите, что для нильпотентности линейного оператора необходимо и достаточно, чтобы он был квазискалярным, с единственным собственным значением $0$.}\\

    \textit{Уточнение. }Для тех, кто не читал методичку Тыртыша (я вот например яйцо читать предпочитаю), квазискалярный это оператор $A: V \rightarrow V$ с единственным собственным значением алгебраической кратности $n = dimV$\\

    \textit{Необходимость:} Если оператор нильпотентный, то есть $\exists  q > 0: A^q = \mathcal{O}$, то для любого его собственного значения $\lambda_i \in \{\lambda_1, \cdots, \lambda_n\}$ (а так как это корни характеристического многочлена, то всегда есть поле --- расширение исходного, в котором будет ровно $n$ собственных значений) верно, что $A^qx = \lambda_i^qx = 0$, а так как $x \neq 0$, то $\lambda_i = 0$, таким образом у $A$ есть только одно собственное значение $\lambda = 0$ алгебраической кратности $n$, то есть $A$ --- квазискалярный.\\

    \textit{Достаточность:} Так как оператор $A$ --- квазискалярный, то его характеристический многочлен равен $f(\lambda)|A - \lambda I| = (-\lambda)^n$. По теореме Кэли-Гамильтона $f(A) = (-A)^n = 0 \Rightarrow A^n = 0$, а значит оператор $A$ --- нильпотентный. Ч.Т.Д.
    \item \textbf{Докажите, что линейный оператор  $A$ является квазискалярным с единственным собственным значением $\lambda$, тогда и только тогда, когда сдвинутый оператор $A - \lambda I$ является нильпотентным.}\\

    Пусть $\lambda'$ --- собственное значение оператора $A$, а $x$ --- отвечающий ему собственный вектор. Найдём собственные вектора и значения сдвинутого оператора $A - \lambda I$: $(A - \lambda I) x = Ax - \lambda x = (\lambda' - \lambda)x$, то есть собственный вектор оператора $A$ будет и собственным вектором оператора $A - \lambda I$, а отвечающее ему собственное значение будет равно $\Lambda =  \lambda' - \lambda$. Так как оператор $A$ является в свою очередь сдвинутым для оператора $A - \lambda I$, то значит собственные вектора обоих операторов совпадают, а собственные значения связаны соотношением $\lambda = \Lambda + \lambda'$\\

    Если оператор $A - \lambda I$ нильпотентный, то в силу предыдущей задачи он квазискалярный с единственным собственным значением $\Lambda = 0$, значит собственное значение оператора $A$ единственное и равно $\Lambda + \lambda = \lambda$ кратности $n$, то есть оператор $A$ --- квазискалярный с единственным собственным значением $\lambda$.\\

    Справедливо и обратное, если $A$ --- квазискалярный с единственным собственным значением $\lambda$, то сдвинутый оператор $A - \lambda I$ имеет единственное собственное значение $\Lambda = \lambda - \lambda = 0$ кратности $n$, то есть  $A - \lambda I$ --- квазискалярный оператор с единственным собственным значением $\Lambda = 0$, а значит он нильпотентный. Ч.Т.Д.
    
    \item \textbf{Докажите, что любой вырожденный оператор либо является нильпотентным, либо расщепляется в сумму нильпотентного и обратимого операторов.}
    \item \textbf{Пусть линейный оператор действует на конечномерном пространстве над полем, которое содержит все корни его характеристического многочлена. докажите, что он расщепляется на прямую сумму своих сужений на корневые подпространства своих попарно различных собственных значений.}
    \item \textbf{Матрица $A$ порядка $n$ имеет попарно различные собственные значения $\lambda_1, \cdots ,\lambda_n$ и соответствующие им собственные векторы $v_1, \cdots, v_n $. Найти собственные векторы линейного оператора $X \rightarrow A^3XA^4, \quad X \in \mathds{C}^{n \times n}$.}
    \item \textbf{Докажите, что минимальное инвариантное относительно оператора $A$ подпространство $M(A, x)$, содержащее заданный ненулевой вектор $x$, совпадает с пространством Крылова $L_k(A, x)$, содержащим вектор $A^kx$. Его размерность равна минимальному значению $k$, при котором $A^kx \in L_k(A, x)$}
    \item \textbf{Докажите, что минимальное инвариантное подпространство относительно оператора $A$ --- $M(A, x)$, содержащее заданный ненулевой вектор $x$, нерасщепляемо тогда и только тогда, когда сужение оператора $A$ на нём квазискалярно.}
    \item \textbf{Докажите, что если $B$ --- нильпотентный оператор, векторы $x, Bx, \cdots, B^{k - 1}x$ ненулевые, а $B^kx = 0$, то $x, Bx, \cdots, B^{k - 1}x$ векторы линейно независимы.}
    \item \textbf{Линейный оператор $A$ называется нильпотентным на векторе $x \neq 0$, если существует натуральное число $k$, для которого $A^kx = 0$. Минимальное такое $k$ называется индексом нильпотентности оператора $A$  на векторе $x$. Пусть $A$ --- линейный оператор и $k_1, \cdots, k_t$ --- его индексы нильпотентности на векторах $x_1, \cdots, x_t$. Докажите, что для линейной независимости составной системы векторов Крылова $x_1, Ax_1, \cdots, A^{k_1 - 1}x_1, \cdots , x_t, Ax_t, \cdots, A^{k_t - 1}x_t$ необходима и достаточна линейная независимость векторов $A^{k_1 - 1}x_1, \cdots, A^{k_t - 1}x_t$.}
    \item \textbf{докажите, что любой нильпотентный оператор, действующий на конечномерном пространстве, расщепляется в прямую сумму нерасщепляемых операторов, действущих на инвариантных подпространствах Крылова.}
    \item \textbf{Докажите, что в любом расщеплении конечномерного пространства в прямую сумму инвариантных подпространств Крылова для нильпотентного оператора $A$ число подпространств размерности $k$ равно $N_k= 2defA^k - defA^{k - 1} - defA^{k + 1}$.}
    \item \textbf{Докажите, что максимальное расщепление линейного оператора $A$ на конечномерном пространстве состоит из квазискалярных операторов, отвечающих его попарно различным собственным значениям и действующих на инвариантных пространствах Крылова. При этом число подпространств размерности $k$ для собственного значения $\lambda$ равно: $N_k(\lambda) = 2def(A - \lambda I)^k - def(A - \lambda I)^{k - 1} - def(A - \lambda I)^{k + 1}$.}
    \item \textbf{Докажите, что любая комплексная матрица $A$ подобна некоторой жордановой матрице. При этом в любой жордановой матрице $A$ число жордановых клеток вида $J(\lambda, k)$ равно $N_k(\lambda) = 2def(A - \lambda I)^k - def(A - \lambda I)^{k - 1} - def(A - \lambda I)^{k + 1}$.}
    \item \textbf{Докажите, что комплексные матрицы подобны в том и только в том случае, когда они имеют одну и ту же жорданову форму с точностью до  перестановки жордановых клеток.}
    \item \textbf{Всегда ли можно построить жорданов базис, содержащий произвольно выбранные базисы в собственных подпространствах.}
    \item \textbf{Докажите, что любая вещественная матрифа порядка $n \geqslant 2$ имеет вещественное инвариантное подпространство размерности 2.}
    \item \textbf{Нильпотентная матрица $J$ порядка $n = 10$ имеет две жордановы клетки порядка $3$ и две жордановы клетки порядка $2$. Найдите жорданову форму матрицы $A = J^2$.}
    \item \textbf{Известно, что $A^k = A, \quad k \geqslant 2$. Докажите, что матрица $A$ диагонализуема.}
    \item \textbf{Сумма двух линейных операторов, действующих на конечномерном пространстве $V$, является обратимым оператором, а произведение --- нулевым оператором. Докажите, что сумма рангов этих операторов равна размерности пространства $V$.}
    \item \textbf{Матрицы $A, B, C, X, Y$ порядка $n$ над одним и тем же полем удовлетворяют равенству $XA + BY = C$. Докажите эквивалентность матриц $L = 
    \begin{bmatrix}
        A & 0\\
        C & B\\
    \end{bmatrix}
    $ и $D = 
    \begin{bmatrix}
        A & 0\\
        0 & B\\
    \end{bmatrix}
    $.}
    \item \textbf{Многочлены $f_A(\lambda) = \prod_{i=1}^{n}(\lambda - \lambda_i)$ и $f_B(\lambda) = \prod_{i=1}^{n}(\lambda - \mu)$ являются характеристическими многочленами матриц $A$ и $B$. Найти характеристический многочлен линейного оператора $X \rightarrow AX + XB$, действующего на пространстве $n \times n$ матриц $X$.}
    \item \textbf{Пусть $A$ --- матрица порядка $n$ и ранга $r$. Докажите, что матрица $I + A$ имеет собственное значение $1$ кратности $\geqslant n - r$.}
    \item \textbf{Все элементы вещественной квадратной матрицы неотрицательны, а суммы элементов в каждой строке одинаковы и равны $\alpha$. докажите, что число $\alpha$ является наибольшим по модулю собственным значением данной матрицы.}
    \item \textbf{комплексные квадратные матрицы $A_1, \cdots , A_m$ попарно коммутируют. Докажите, что они имеют общий собственный вектор.}
    \item \textbf{Докажите, что если $A \in \mathds{R}^{n \times n}$ и $rank(I - A) = n - rank(A)$, то $A^2 = A$.}
    \item \textbf{линейный оператор $A$ на конечномерном пространстве удовлетворяет условию $A^2 =A$. Найдите его собственные значения и докажите, что он имеет простую структуру.}
    \item \textbf{Найдите жорданову форму матрицы $A^3$, где $A$ --- жорданова клетка порядка $n$.}
    \item \textbf{Найдите собственные значения и собственные подпространства комплексной матрицы $A = I - 2uu^{*}$, где $u^{*}u = 1, \quad u \in \mathds{C}^n$.}
    \item \textbf{В вещественной матрице, которая является нильпотентной жордановой клеткой порядка $n$, элемент в позиции $(n, 1)$ заменяется на $\varepsilon > 0$. Докажите, что все собственные значения возмущённой матрицы простые и по модулю равны $\varepsilon^{\frac{1}{n}}$.}
    \item \textbf{Докажите, что линейный оператор $A: V \rightarrow V$ коммутирует со всеми линейными операторами, действующими на бесконечномерном пространстве $V$ над полем $\mathds{P}$, в том и только в том случае, когда он является оператором умножения на некоторое число из поля $\mathds{P}$.}
    \item \textbf{Пусть $A$ и $B$ --- квадратные матрицы над одним и тем же полем. Докажите, что характеристические многочлены матриц $AB$ и $BA$ совпадают.}
    \item \textbf{Пусть верхнетреугольная матрица порядка $n = n_1 + n_2$ имеет вид $
    \begin{bmatrix}
        A_{11} & A_{12}\\
        0 & A_{22}\\
    \end{bmatrix}
    $ и при этом блоки $A_{11} \in \mathds{C}^{n_1 \times n_1}$ и $A_{22} \in \mathds{C}^{n_2 \times n_2}$ не имеют общий собственных значений. Докажите, что существует матрица $X \in \mathds{C}^{n_1 \times n_2}$ такая, что
    \[
    \begin{bmatrix}
        I & X\\
        0 & I\\
    \end{bmatrix}
    \begin{bmatrix}
        A_{11} & A_{12}\\
        0 & A_{22}\\
    \end{bmatrix}
    \begin{bmatrix}
        I & X\\
        0 & I\\
    \end{bmatrix}^{-1}
    =
    \begin{bmatrix}
        A_{11} & 0\\
        0 & A_{22}\\
    \end{bmatrix}
    \]
    }
    \item \textbf{Пусть $J$ --- жорданова клетка порядка $n$ с нулевым собственным значением. Докажите, что уравнение $X^2 = J$ относительно $X \in \mathds{C}^{n \times n}$ не имеет решений, если $n \geqslant 2$.}
    \item \textbf{докажите, что любая вещественная матрица с помощью вещественного преобразования подобия приводится к прямой сумме вещественных жордановых клеток и вещественных  блочных жордановых клеток.}
    \item \textbf{докажите, что матрица порядка $n \geqslant 2$ имеет конечное число инвариантных подпространств в том и только в том случае, когда каждому собственному значению соответствует ровно одна жорданова клетка.}
    \item \textbf{ Пусть $V, W ,Z$ --- конечномерные линейные пространства над общим полем. Докажите, что для произвольных линейных операторов $B : V \rightarrow Z$ и $A : Z \rightarrow W$ выполнено неравенство $rankA  + rankB - rank(AB) \leqslant dimZ$.}
    \item \textbf{Докажите, что вещественная квадратная матрица подобна своей транспонированной матрице. Верно ли это для матриц над произвольным полем?}
    \item \textbf{Докажите, что диагонализуемая матрица ранга $r$ обладает ненулевым главным минором порядка $r$.}
    \item \textbf{Докажите, что собственные значения матрицы $A = [a_{kl}]$ с элементами $a_{kl} = \varepsilon^{(k - 1)(l - 1)}, 1 \leqslant k, l \leqslant n$, где $\varepsilon$--- первообразный корень степени $n$ из единицы, равны $\pm n^{\frac{1}{2}}$ и $\pm in^{\frac{1}{2}}$.}
    \item \textbf{опишите алгоритм вычисления коэффициентов характеристического многочлена верхней почти треугольной матрицы порядка $n$ с числом арифметических операций $O(n^3)$.}
    \item \textbf{Пусть квадратная матрица $A$ с элементами из некоторого поля аннулируется неприводимым многочленом с коэффициентами из этого поля. Докажите, что если степень $n$ этого многочлена равна порядку матрицы, то система векторов Крылова $x, Ax, \cdots, A^{n-1}x$ будет линейно независимой для какого-то вектора $x$ с элементами из этого поля.}
\end{enumerate}
\section{Расстояния, нормы, скалярные произведения, полиэдры}
\begin{enumerate}%задачки оформляем так
    \item \textbf{текст очередной задачи}\\

    Решение/доказательство
    \item \textbf{текст ещё одной задачи}%и так далее



задача №67 по запросу Зизова:

Интеграл $a_{ij} = \int \limits_{0}^{1} x^{i + j} \rho(x)dx = \xi_{i + j}^{i + j} \int \limits_{0}^{1} \rho(x)dx = c \xi_{i + j}^{i + j}$ , где $c = \int \limits_{0}^{1} \rho(x)dx$ по первой теореме о среднем. Тогда можно записать матрицу оператора:
\[
A = c
\begin{bmatrix}
    \xi_2^2 & \xi_3^3 & \cdots & \xi_{n + 1}^{n + 1}\\
    \xi_3^3 & \xi_4^4 & \cdots & \xi_{n + 2}^{n + 2}\\
    \vdots & \vdots & \ddots & \vdots \\
    \xi_{n + 1}^{n + 1} & \xi_{n + 2}^{n + 2} & \cdots & \xi_{2n}^{2n}\\
\end{bmatrix}
\]

\end{enumerate}
\section{Нормальные операторы, эрмитовы матрицы}
\begin{enumerate}
    \item 
\end{enumerate}
\newpage
\begin{center}
    \textbf{Post Scriptum}
\end{center}
\begin{flushright}
    "Оставь Надежду, всяк сюда входящий"\\
    ауфная нацистская цитата с ворот Освенцима 
\end{flushright}

Парни и тяночки, здесь и далее линала больше не будет, а будет кринж, мои мысли, мысли моих коллег(калек, ахах(привыкайте к скобочкам, я их люблю)) и, разумеется еврейские анекдоты. Во-первых, хочу сказать спасибо всем, кто помогал мне с этой ПДФкой, а также всей моей группе, вы няшки. Во-вторых, спасибо однокурсникам за то, что вы были и были адекватны и за то, что со своими шутками я до сих пор жив. В-третьих, если вы тян (ну или милый и ласковый мальчик), то я не обижусь, если мне на почту вы будете писать не только с правками и вопросами.\\

Приходит Изя в публичный дом в Иерусалиме, спрашивает, есть ли сегодня Розочка. Ему говорят, что да, есть, 200\$ стоит. Снял он Розочку, сделал дело, заплатил. На следующий день прриходит --- та же история. И на третий день снова приходит к Розочке, а она и говорит Изе: "что вы это, Изенька, всё ко мне ходите, вот уже третий день, вы что, влюбились в меня?!". "Да ну что вы, Розочка" --- отвечает ей Изя: "меня просто ваша тетя Сара из Одессы просила вам 600\$ передать". 

\end{document}
