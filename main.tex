%моя любимая преамбула
\documentclass[12pt]{article} %документ с 12 шрифтом
\usepackage{blindtext} %чтобы делать lorem ipsum
\usepackage{xcolor} %цветной текст
\usepackage[a4paper, left = 2cm, right = 2cm, top = 1cm]{geometry}%геометрия документа (отступы короче)
\usepackage{graphicx} % Required for inserting images
\usepackage[utf8x]{inputenc} % Включаем поддержку UTF8  
\usepackage[russian]{babel} %поддержка русского языка
\usepackage{amsmath} %пакеты для некоторых математических символов
\usepackage{ amssymb }
\usepackage{ dsfont } %красивые буковки с двойными черточками, как для множества R
\usepackage{physics} %и снова символы (тех считает, что самые красивые греческие буквы нужны физикам)
\usepackage[normalem]{ulem}


\begin{document}

\begin{titlepage}
    \begin{center}
  
       \vspace*{1cm}
        \huge
       \textbf{Вопросики по линальчику}
        \normalsize

  
            
       \vspace{1.5cm}

       \begin{flushright}
       Сделали n неравнодушных шизов:\\
       Владимир Якунин \\
       (waiting for answer)Надежда Белоусова\\
       (waiting for commits) Ирина клочкова\\
       (waiting for commits) Арсений Поздняков\\
       \end{flushright}

       \vfill
            тут могла быть ваша реклама
       \vfill
     
        если у вас возникли вопросики, пишите сюда YakuninVla@yandex.ru
        
       МГУ, Факультет ВМиК\\
       Россия\\
       Май 2023
    \end{center}
\end{titlepage}

\tableofcontents

\section{Линейные операторы, инвариантность, спектр}

\begin{enumerate}
    \item \textbf{Докажите, что линейная комбинация линейных операторов является линейным оператором.}\\
    
    Рассмотрим оператор $A = \gamma_1 A_1 + \cdots + \gamma_n A_n$ и проверим, является ли он линейным:
    \[
    A(\alpha x + \beta y) = (\gamma_1 A_1 + \cdots + \gamma_n A_n)(\alpha x + \beta y) = \gamma_1 A_1 \alpha x + \cdots + \gamma_n A_n \alpha x + \gamma_1 A_1 \beta y + \cdots + \gamma_n A_n \beta y = \]\[
    = \alpha \gamma_1 A_1  x + \cdots + \alpha \gamma_n A_n  x + \beta \gamma_1 A_1  y + \cdots + \beta \gamma_n A_n  y =\]\[ \alpha(\gamma_1 A_1 + \cdots + \gamma_n A_n)x + \beta(\gamma_1 A_1 + \cdots + \gamma_n A_n)y = \alpha A x + \beta A y
    \]
    таким образом оператор $A$ --- линейный, чтд.
    \item \textbf{Докажите, что произведение линейных операторов является линейным оператором.}\\
    
     Пусть есть линейные операторы $A: V \rightarrow W$ и $B : W \rightarrow Z$ .Рассмотрим оператор $C : V \rightarrow Z$, $C = BA$ и проверим, является ли он линейным:
     \[
     C(\alpha x + \beta y) = BA(\alpha x + \beta y) = BA\alpha x + BA \beta y = B (\alpha A x) + B(\beta A  y) = \alpha BA x + \beta AB y = \alpha C x + \beta C y
     \]
     как видим, оператор $C$ --- линейный.
    \item \textbf{Докажите, что если линейный оператор обратим, то обратный к нему определён однозначно и является линейным оператором.}\\
    
    пусть $A : V \rightarrow W$ --- обратимый линейный оператор и $B : W \rightarrow V$ --- обратный оператор к $A$: $BA = I = ABAB = AB$.  \\
    Единственность: пусть есть оператор $C \neq B$ такой что $AC = CA = I$, тогда 
    \[
    0 = I - I = AB - AC = A(B - C)
    \]
    значит либо $A = 0$, либо $B - C = 0 \Rightarrow B = C$.\\
    Линейность: \[
    B(\alpha x + \beta y) = B(\alpha AB x + \beta AB y) = B(A(\alpha B x + \beta B y)) = \alpha B x + \beta B y
    \]
    несложно увидеть, что оператор $B$ --- линейный. дальше обозначаем по-нормальному $B = A^{-1}$.
    
    \item \textbf{Пусть $V$ и $W$ --- конечномерные пространства над общим полем. Докажите, что для обратимости линейного оператора $A: V \rightarrow W$ необходимо и достаточно выполнение условий $dimV = dimW$ и $kerA = 0$.}\\

    Необходимость: Так как для $\forall x \in V$ существует прообраз $y = Ax \in W$, ведь $x = A^{-1}y$, то $dimV \leqslant dimW$. Так как для $\forall y \in W$ существует прообраз $x = A^{-1}y \in V$, ведь $y = Ax$, то $dimW \leqslant dimV$. А значит $dimV = dimW$. Теперь пусть существует $x \in V, \quad x \neq 0$ принадлежащий ядру оператора $A$: $Ax = 0$. Тогда $A^{-1}0 = x$, но так как $A^{-1}$ --- линейный, то $A^{-1}0 = 0 \neq x$ --- противоречие.\\

    Достаточность: пусть $e_1, \cdots, e_n$ --- базис $V$, тогда если $x = \alpha_1 e_1 + \cdots + \alpha_n e_n$, то $y = Ax = \alpha_1 Ae_1 + \cdots + \alpha_n Ae_n$, если вектора $Ae_1, \cdots, Ae_n$ --- линейно зависимы, то существует их нетривиальная линейная комбинация, равная нулю: $\beta_1Ae_1 + \cdots + \beta_nAe_n = 0 \Rightarrow A(\beta_1 e_1 + \cdots + \beta_n e_n) = 0$, однако так как $e_1, \cdots, e_n$ линейно независимы, то $z = \beta_1 e_1 + \cdots + \beta_n e_n \neq 0$, но это значит, что $\exists z \in V, z \neq 0 : Az = 0 \Rightarrow kerA \neq 0$ --- противоречие, значит $Ae_1, \cdots, Ae_n$ линейно независимы и их число равно размерности $W$, а значит это базис $W$.\\
    Теперь определим действие оператора $A^{-1}$ на базисных векторах $W$ и покажем, что он будет обратным к $A$:
    $A^{-1}(Ae_k) = e_k \quad \forall k \in \{1, \cdots, n\}$ такой оператор существует и единственный (потому что для того, чтобы задать оператор, достаточно задать его действие на базисных векторах области определения), и при этом $A^{-1}Ax = A^{-1}(\alpha_1 Ae_1 + \cdots + \alpha_n Ae_n) = \alpha_1 A^{-1} Ae_1 + \cdots + \alpha_n A^{-1} Ae_n = \alpha_1 e_1 + \cdots + \alpha_n e_n = x$, значит $A^{-1}$ --- обратный оператор. Ч.Т.Д.
    \item \textbf{Докажите, что ядро и образ линейного оператора являются его инвариантными подпространствами.}\\

    \textit{Ядро}: Пусть $A: V \rightarrow V$, $x \in  V,\quad Ax = 0$, то есть $x \in kerA$, тогда $Ax = 0 \in kerA \quad \forall x \in kerA$\\

    \textit{образ}: Пусть $A: V \rightarrow V$, $y \in imA$, то есть $\exists x: Ax = y$, тогда $Ay = z \in imA$, так как у $z$ есть прообраз $y$.
    
    \item \textbf{Докажите, что сумма ранга и дефекта линейного оператора равна размерности его области определения.}\\

    Ядро линейного оператора $A: V \rightarrow W$ --- линейное подпространство пространства $V$, базис ядра: $a_1, \cdots, a_k$, размерность ядра $dim(kerA) = k$. дополним его до базиса пространства $V$: $a_1, \cdots, a_k, b_{k + 1}, \cdots, b_n$. Заметим, что так как $b_{k + 1}, \cdots, b_n \in V \setminus kerA$, то $Ab_i \neq 0 \forall i \in \{k + 1, \cdots, n\}$, вектора $Ab_{k + 1}, \cdots, Ab_n$ --- линейно независимы.\\ Действительно, если бы они были бы линейно зависимы, то существовала бы нетривиальная линейная комбинация $\alpha_1 Ab_{k + 1} + \cdots + \alpha_{n - k} A b_n = 0 \Rightarrow A(\alpha_1 b_{k + 1} + \cdots + \alpha_{n - k} b_n) = 0$, то есть $\alpha_1 b_{k + 1} + \cdots + \alpha_{n - k} b_n \in kerA$, а это протиоречит выбору векторов $b_{k + 1}, \cdots, b_n$. Теперь разложим $\forall x \in V$ в выбранном базисе, подействуем на него оператором и посмотрим, что интересного произойдёт: $Ax = \alpha_1 Aa_1 + \cdots + \alpha_k Aa_k + \alpha_{k + 1}Ab_{k + 1} + \cdots + \alpha_nAb_n = \alpha_{k + 1}Ab_{k + 1} + \cdots + \alpha_nAb_n$, заметим, что любой вектор из образа можно представить как линейную комбинацию $n - k$ линейно независимых векторов $Ab_{k + 1} + \cdots + Ab_n$, то есть это базис образа, то есть $dim(imA) = n - k$. Осталось сложить дважды два: $dim(kerA) + dim(imA) = k + (n - k) = n$. Ч.Т.Д.
    
    \item \textbf{Докажите, что сумма ядер двух линейных операторов, действующих на одном пространстве, совпадает с этим пространством, то образ суммы равен сумме образов.}
    \item \textbf{Линейный оператор $A: V \rightarrow V$ удовлетворяет равенству $A^m = 0$. Докажите, что оператор $I - A$ --- обратим.}
    \item \textbf{Линейные операторы $A$ и $B$ таковы, что оператор $A + B$ обратимый. Докажите, что операторы $P = (A + B)^{-1}A$ и $Q = (A + B)^{-1}B$ коммутируют.}
    \item \textbf{Докажите, что для того, чтобы оператор $P: V \rightarrow V$ был оператором проектирования, необходимо и достаточно, чтобы выполнялось условие $P^2 = P$.}
    \item \textbf{Докажите, что для того чтобы матрицы одного размера были матрицами одного и того же линейного оператора в каких-то арах базисов необходимо и достаточно, чтобы они имели одинаковый ранг.}
    \item \textbf{Докажите, что определитель и след квадратной матрицы являются инвариантами подобия}
    \item \textbf{Докажите, что барактеристический многочлен квадратной матрицы является инвариантом подобия.}
    \item \textbf{Найдите характеристический многочлен матрицы $A = 
    \begin{bmatrix}
         &&1\\
         &\rotatebox[origin=c]{75}{$\ddots$}&\\
         1&& \\
    \end{bmatrix}_{n \times n}
    $}.
    \item \textbf{Найдите все инвариантные подпространства оператора дифференциирования в пространстве всех вещественных многочленов.}
    \item \textbf{Докажите, что число является собственным значением линейного оператора на конечномерном пространстве в том и только в том случае, когда оно является корнем его характеристического многочлена.}
    \item \textbf{линейный оператор действует в $n$-мерном пространстве над полем, содержащем все корни его характеристического многочлена. Докажите, что существует базис пространства , в котором матрица оператора имеет верхний треугольный вид с главной диагональю, заполненной корнями характеристического многочлена в любом заранее заданном порядке.}
    \item \textbf{Докажите, что если матрицы $A$ и $B$ подобны, то для произвольного многочлена $f(\lambda)$ матрицы $f(A)$ и $f(B)$ тоже подобны.}
    \item \textbf{Докажите, что минимальный многочлен, аннулирующий квадратную матрицу, является делителем её характеристического многочлена.}
    \item \textbf{докажите, что любая квадратная матрица с элементами из произвольного поля аннулируется своим характеристическим многочленом.}
    \item \textbf{Докажите, что любой приведённый многочлен степени выше первой является характеристическим многочленом некоторой матрицы.}
    \item \textbf{Преобразование $A \rightarrow PAP^{-1}$ будем называть элементарным преобразованием подобия, если матрица $P$ является матрицей перестановки, либо матрицей вида $P = I + \gamma E_{ij}$, где матрица $E_{ij} $ отличается от нулевой только единицей в позиции $(i, j)$, при $i \neq j$, а число $\gamma$ --- произвольное. Докажите, что с помощью $O(n^2)$ элементарных преобразований подобия матрицу порядка $n$ можно привести к верхнему почти треугольному виду}%ура, путём долгих раздумий я понял, почему не просто к треугольному виду, там же может быть верхняя строка из нулей.
    \item \textbf{Докажите, что алгебраическая кратность собственного значения не меньше его геометрической кратности.}
    \item \textbf{Докажите, что собственные векторы попарно различных собственных значений линейно независимы.}
    \item \textbf{докажите, что если матрица порядка $n$ имеет $n$ собственных значений, то она диагонализуема.}
    \item \textbf{Докажите, что для нильпотентности линейного оператора необходимо и достаточно, чтобы он был квазискалярным, с собственным значением $0$.}
    \item \textbf{Докажите, что линейный оператор  $A$ является квазискалярным с елинственным собственным значением $\lambda$, если сдвинутый оператор $A - \lambda I$ является нильпотентным.}
    \item \textbf{Докажите, что любой вырожденный оператор либо является нильпотентным, либо расщепляется в сумму нильпотентного и обратимого операторов.}
    \item \textbf{Пусть линейный оператор действует на конечномерном пространстве над полем, которое содержит все корни его характеристического многочлена. докажите, что он расщепляется на прямую сумму своих сужений на корневые подпространства своих попарно различных собственных значений.}
    \item \textbf{Матрица $A$ порядка $n$ имеет попарно различные собственные значения $\lambda_1, \cdots ,\lambda_n$ и соответствующие им собственные векторы $v_1, \cdots, v_n $. Найти собственные векторы линейного оператора $X \rightarrow A^3XA^4, \quad X \in \mathds{C}^{n \times n}$.}
    \item \textbf{Докажите, что минимальное инвариантное относительно оператора $A$ подпространство $M(A, x)$, содержащее заданный ненулевой вектор $x$, совпадает с пространством Крылова $L_k(A, x)$, содержащим вектор $A^kx$. Его размерность равна минимальному значению $k$, при котором $A^kx \in L_k(A, x)$}
    \item \textbf{Докажите, что минимальное инвариантное подпространство относительно оператора $A$ --- $M(A, x)$, содержащее заданный ненулевой вектор $x$, нерасщепляемо тогда и только тогда, когда сужение оператора $A$ на нём квазискалярно.}
    \item \textbf{Докажите, что если $B$ --- нильпотентный оператор, векторы $x, Bx, \cdots, B^{k - 1}x$ ненулевые, а $B^kx = 0$, то $x, Bx, \cdots, B^{k - 1}x$ векторы линейно независимы.}
    \item \textbf{Линейный оператор $A$ называется нильпотентным на векторе $x \neq 0$, если существует натуральное число $k$, для которого $A^kx = 0$. Минимальное такое $k$ называется индексом нильпотентности оператора $A$  на векторе $x$. Пусть $A$ --- линейный оператор и $k_1, \cdots, k_t$ --- его индексы нильпотентности на векторах $x_1, \cdots, x_t$. Докажите, что для линейной независимости составной системы векторов Крылова $x_1, Ax_1, \cdots, A^{k_1 - 1}x_1, \cdots , x_t, Ax_t, \cdots, A^{k_t - 1}x_t$ необходима и достаточна линейная независимость векторов $A^{k_1 - 1}x_1, \cdots, A^{k_t - 1}x_t$.}
    \item \textbf{докажите, что любой нильпотентный оператор, действующий на конечномерном пространстве, расщепляется в прямую сумму нерасщепляемых операторов, действущих на инвариантных подпространствах Крылова.}
    \item \textbf{Докажите, что в любом расщеплении конечномерного пространства в прямую сумму инвариантных подпространств Крылова для нильпотентного оператора $A$ число подпространств размерности $k$ равно $N_k= 2defA^k - defA^{k - 1} - defA^{k + 1}$.}
    \item \textbf{Докажите, что максимальное расщепление линейного оператора $A$ на конечномерном пространстве состоит из квазискалярных операторов, отвечающих его попарно различным собственным значениям и действующих на инвариантных пространствах Крылова. При этом число подпространств размерности $k$ для собственного значения $\lambda$ равно: $N_k(\lambda) = 2def(A - \lambda I)^k - def(A - \lambda I)^{k - 1} - def(A - \lambda I)^{k + 1}$.}
    \item \textbf{Докажите, что любая комплексная матрица $A$ подобна некоторой жордановой матрице. При этом в любой жордановой матрице $A$ число жордановых клеток вида $J(\lambda, k)$ равно $N_k(\lambda) = 2def(A - \lambda I)^k - def(A - \lambda I)^{k - 1} - def(A - \lambda I)^{k + 1}$.}
    \item \textbf{Докажите, что комплексные матрицы подобны в том и только в том случае, когда они имеют одну и ту же жорданову форму с точностью до  перестановки жордановых клеток.}
    \item \textbf{Всегда ли можно построить жорданов базис, содержащий произвольно выбранные базисы в собственных подпространствах.}
    \item \textbf{Докажите, что любая вещественная матрифа порядка $n \geqslant 2$ имеет вещественное инвариантное подпространство размерности 2.}
    \item \textbf{Нильпотентная матрица $J$ порядка $n = 10$ имеет две жордановы клетки порядка $3$ и две жордановы клетки порядка $2$. Найдите жорданову форму матрицы $A = J^2$.}
    \item \textbf{Известно, что $A^k = A, \quad k \geqslant 2$. Докажите, что матрица $A$ диагонализуема.}
    \item \textbf{Сумма двух линейных операторов, действующих на конечномерном пространстве $V$, является обратимым оператором, а произведение --- нулевым оператором. Докажите, что сумма рангов этих операторов равна размерности пространства $V$.}
    \item \textbf{Матрицы $A, B, C, X, Y$ порядка $n$ над одним и тем же полем удовлетворяют равенству $XA + BY = C$. Докажите эквивалентность матриц $L = 
    \begin{bmatrix}
        A & 0\\
        C & B\\
    \end{bmatrix}
    $ и $D = 
    \begin{bmatrix}
        A & 0\\
        0 & B\\
    \end{bmatrix}
    $.}
    \item \textbf{Многочлены $f_A(\lambda) = \prod_{i=1}^{n}(\lambda - \lambda_i)$ и $f_B(\lambda) = \prod_{i=1}^{n}(\lambda - \mu)$ являются характеристическими многочленами матриц $A$ и $B$. Найти характеристический многочлен линейного оператора $X \rightarrow AX + XB$, действующего на пространстве $n \times n$ матриц $X$.}
    \item \textbf{Пусть $A$ --- матрица порядка $n$ и ранга $r$. Докажите, что матрица $I + A$ имеет собственное значение $1$ кратности $\geqslant n - r$.}
    \item \textbf{Все элементы вещественной квадратной матрицы неотрицательны, а суммы элементов в каждой строке одинаковы и равны $\alpha$. докажите, что число $\alpha$ является наибольшим по модулю собственным значением данной матрицы.}
    \item \textbf{комплексные квадратные матрицы $A_1, \cdots , A_m$ попарно коммутируют. Докажите, что они имеют общий собственный вектор.}
    \item \textbf{Докажите, что если $A \in \mathds{R}^{n \times n}$ и $rank(I - A) = n - rank(A)$, то $A^2 = A$.}
    \item \textbf{линейный оператор $A$ на конечномерном пространстве удовлетворяет условию $A^2 =A$. Найдите его собственные значения и докажите, что он имеет простую структуру.}
    \item \textbf{Найдите жорданову форму матрицы $A^3$, где $A$ --- жорданова клетка порядка $n$.}
    \item \textbf{Найдите собственные значения и собственные подпространства комплексной матрицы $A = I - 2uu^{*}$, где $u^{*}u = 1, \quad u \in \mathds{C}^n$.}
    \item \textbf{В вещественной матрице, которая является нильпотентной жордановой клеткой порядка $n$, элемент в позиции $(n, 1)$ заменяется на $\varepsilon > 0$. Докажите, что все собственные значения возмущённой матрицы простые и по модулю равны $\varepsilon^{\frac{1}{n}}$.}
    \item \textbf{Докажите, что линейный оператор $A: V \rightarrow V$ коммутирует со всеми линейными операторами, действующими на бесконечномерном пространстве $V$ над полем $\mathds{P}$, в том и только в том случае, когда он является оператором умножения на некоторое число из поля $\mathds{P}$.}
    
\end{enumerate}
\section{Расстояния, нормы, скалярные произведения, полиэдры}
\begin{enumerate}%задачки оформляем так
    \item \textbf{текст очередной задачи}\\

    Решение/доказательство
    \item \textbf{текст ещё одной задачи}%и так далее
\end{enumerate}
\section{Нормальные операторы, эрмитовы матрицы}
\begin{enumerate}
    \item 
\end{enumerate}
\newpage
\begin{center}
    \textbf{Post Scriptum}
\end{center}
\begin{flushright}
    "Оставь Надежду, всяк сюда входящий"\\
    ауфная нацистская цитата с ворот Освенцима 
\end{flushright}

Парни и тяночки, здесь и далее линала больше не будет, а будет кринж, мои мысли, мысли моих коллег(калек, ахах(привыкайте к скобочкам, я их люблю)) и, разумеется еврейские анекдоты. Во-первых, хочу сказать спасибо всем, кто помогал мне с этой ПДФкой, а также всей моей группе, вы няшки. Во-вторых, спасибо однокурсникам за то, что вы были и были адекватны и за то, что со своими шутками я до сих пор жив. В-третьих, если вы тян (ну или милый и ласковый мальчик), то я не обижусь, если мне на почту вы будете писать не только с правками и вопросами.\\

Приходит Изя в публичный дом в Иерусалиме, спрашивает, есть ли сегодня Розочка. Ему говорят, что да, есть, 200\$ стоит. Снял он Розочку, сделал дело, заплатил. На следующий день прриходит --- та же история. И на третий день снова приходит к Розочке, а она и говорит Изе: "что вы это, Изенька, всё ко мне ходите, вот уже третий день, вы что, влюбились в меня?!". "Да ну что вы, Розочка" --- отвечает ей Изя: "меня просто ваша тетя Сара из Одессы просила вам 600\$ передать". 

\end{document}
